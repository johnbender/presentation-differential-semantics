\documentclass{beamer}
\usefonttheme{serif}
\usepackage{bbold}
\usepackage{semantic}
\usepackage{palatino}
\usepackage{listings}
\usepackage{mathtools}
\usepackage{xstring}
\usepackage{syntax}
\usepackage{color}

\renewcommand{\syntleft}{\normalfont\itshape}
\renewcommand{\syntright}{}

\newcommand{\hilight}[1]{\colorbox{yellow}{#1}}

\newcommand{\cmp}[2]{
  \mathcal{#1}|[ \mathtt{ \StrSubstitute{#2}{ }{\ } } |]
}

\lstdefinelanguage{diff}{
  morecomment=[f][\color{blue}]{@@},
  morecomment=[f][\color{red}]-,
  morecomment=[f][\color{green}]+,
  morecomment=[f][\color{magenta}]{---},
  morecomment=[f][\color{magenta}]{+++},
}

\definecolor{solarized@base03}{HTML}{002B36}
\definecolor{solarized@base02}{HTML}{073642}
\definecolor{solarized@base01}{HTML}{586e75}
\definecolor{solarized@base00}{HTML}{657b83}
\definecolor{solarized@base0}{HTML}{839496}
\definecolor{solarized@base1}{HTML}{93a1a1}
\definecolor{solarized@base2}{HTML}{EEE8D5}
\definecolor{solarized@base3}{HTML}{FDF6E3}
\definecolor{solarized@yellow}{HTML}{B58900}
\definecolor{solarized@orange}{HTML}{CB4B16}
\definecolor{solarized@red}{HTML}{DC322F}
\definecolor{solarized@magenta}{HTML}{D33682}
\definecolor{solarized@violet}{HTML}{6C71C4}
\definecolor{solarized@blue}{HTML}{268BD2}
\definecolor{solarized@cyan}{HTML}{2AA198}
\definecolor{solarized@green}{HTML}{859900}

\lstset{
  upquote=true,
  columns=fixed,
  tabsize=2,
  extendedchars=true,
  breaklines=true,
  numbersep=5pt,
  rulesepcolor=\color{solarized@base03},
  numberstyle=\tiny\color{solarized@base01},
  basicstyle=\ttfamily,
  keywordstyle=\color{solarized@green},
  stringstyle=\color{solarized@cyan}\ttfamily,
  identifierstyle=\color{solarized@blue},
  commentstyle=\color{solarized@base01},
  emphstyle=\color{solarized@red}
}

% There are many different themes available for Beamer. A comprehensive
% list with examples is given here:
% http://deic.uab.es/~iblanes/beamer_gallery/index_by_theme.html
% You can uncomment the themes below if you would like to use a different
% one:
%\usetheme{AnnArbor}
%\usetheme{Antibes}
%\usetheme{Bergen}
%\usetheme{Berkeley}
%\usetheme{Berlin}
%\usetheme{Boadilla}
\usetheme{boxes}
%\usetheme{CambridgeUS}
%\usetheme{Copenhagen}
%\usetheme{Darmstadt}
%\usetheme{default}
%\usetheme{Frankfurt}
%\usetheme{Goettingen}
%\usetheme{Hannover}
%\usetheme{Ilmenau}
%\usetheme{JuanLesPins}
%\usetheme{Luebeck}
%\usetheme{Madrid}
%\usetheme{Malmoe}
%\usetheme{Marburg}
%\usetheme{Montpellier}
%\usetheme{PaloAlto}
%\usetheme{Pittsburgh}
%\usetheme{Rochester}
%\usetheme{Singapore}
%\usetheme{Szeged}
%\usetheme{Warsaw}

\title{Differential Semantics}

% A subtitle is optional and this may be deleted
\subtitle{}

\author{John Bender}
% - Give the names in the same order as the appear in the paper.
% - Use the \inst{?} command only if the authors have different
%   affiliation.

\date{CS 239, Spring 2014}
% - Either use conference name or its abbreviation.
% - Not really informative to the audience, more for people (including
%   yourself) who are reading the slides online

\subject{Programming Languages and Systems, Computer Science}
% This is only inserted into the PDF information catalog. Can be left
% out.

% If you have a file called "university-logo-filename.xxx", where xxx
% is a graphic format that can be processed by latex or pdflatex,
% resp., then you can add a logo as follows:

% \pgfdeclareimage[height=0.5cm]{university-logo}{university-logo-filename}
% \logo{\pgfuseimage{university-logo}}

% Delete this, if you do not want the table of contents to pop up at
% the beginning of each subsection:
\AtBeginSubsection[]
{
  \begin{frame}<beamer>{Outline}
    \tableofcontents[currentsection,currentsubsection]
  \end{frame}
}


% Let's get started
\begin{document}

\setlength{\abovedisplayskip}{0pt}
\setlength{\belowdisplayskip}{0pt}
\setlength{\abovedisplayshortskip}{0pt}
\setlength{\belowdisplayshortskip}{0pt}

\begin{frame}
  \titlepage
\end{frame}

\begin{frame}{Semantics or Intent}
  \begin{block}{Semantics Vs. Intent}
    Intent is often a limiting factor in reasoning about programs.

    \begin{itemize}
      \item optimizations
      \item useful error messages
      \item static \& dynamic analysis
      \item declarative languages
  \end{itemize}
  \end{block}
\end{frame}

\begin{frame}{Semantics or Intent}
  \begin{block}{Sources of Both}
    Many resources for how programs behave, few for intended behavior.

    \begin{itemize}
      \item program syntax
      \item inferred types \& annotations
      \item concrete \& symbolic execution
      \item tests, specs, contracts
  \end{itemize}
  \end{block}
\end{frame}

\begin{frame}{Semantics or Intent}
  \begin{block}{Constrained View}
    Reasoning about programs is most often done from a standing start.

    \begin{itemize}
      \item single snapshot
      \item programs evolve
      \item state evolves
  \end{itemize}
  \end{block}
\end{frame}

\begin{frame}{Semantics for Diffs}
  \begin{block}{At a High Level}
    The idea is to assign meaning to patches in a way that augments the runtime's understanding of the programmers intent.

    \begin{itemize}
      \item SQL DDL
      \item Dynamic Software Update (DSU)
    \end{itemize}
  \end{block}
\end{frame}

\begin{frame}[fragile]
  \frametitle{SQL DLL}
  \begin{block}{Declarative}
    SQL's data definition language starts with a descriptive approach to objects in a schema.
  \end{block}

  \begin{example}
    \lstinputlisting[language=SQL]{code/foo-table.sql}
  \end{example}
\end{frame}

\begin{frame}[fragile]
  \frametitle{SQL DLL}
  \begin{block}{Declarativish}
    It quickly degenerates and reasoning about the state of the schema goes out with the bathwater.
  \end{block}

  \begin{example}
    \lstinputlisting[language=SQL]{code/foo-table-alter.sql}
  \end{example}
\end{frame}

\begin{frame}[fragile]
  \frametitle{SQL DLL}
  \begin{block}{Diff View}
    Looking at a diff of the table declaration suggests intent.
  \end{block}

  \begin{example}
    \begin{center}
      \begin{minipage}{.44\textwidth}
        \lstinputlisting[language=diff,identifierstyle=\color{black}]{code/foo-table.diff}
      \end{minipage}
      \hfill
      \begin{minipage}{.48\textwidth}
        \lstinputlisting[language=diff,identifierstyle=\color{black}]{code/foo-table-alter.diff}
      \end{minipage}
    \end{center}
  \end{example}
\end{frame}

\begin{frame}[fragile]{DDL Semantics}{Traditional Grammar}
  \begin{grammar}
    <Program> ::= <Declaration>\lit*{;}

    <Declaration> ::= <Declaration>\lit*{;} <Declaration> | <Create> | <Alter> | <Drop>

    <Create> ::= \lit*{create table} <Identifier> \lit*{(} <Column> \lit*{)}

    <Drop> ::= \lit*{drop table} <Identifier>

    <Column> ::= <Column>, <Column> | <Identifier> <Type>

    <Alter> ::= \lit*{alter table} <Identifier> \lit*{add} <Identifier> <Type>
           \alt \lit*{alter table} <Identifier> \lit*{drop} <Identifier>

    <Type> := \lit*{int} | \lit*{varchar(10)}
  \end{grammar}
\end{frame}

\begin{frame}[fragile]{DDL Semantics}{Denotational Semantics}
  \footnotesize
  \begin{alignat*}{2}
    &\mathcal{S} : Declaration -> Schema -> Schem&&a \\
    &\cmp{S}{D;}s &&= \cmp{S}{D}s \\
    &\cmp{S}{D1;D2}s &&= \cmp{S}{D2}\ (\cmp{S}{D1}\ s) \\
    &\cmp{S}{create table I ( C )} s &&=
      schemaChange\ \mathtt{I}\ (\cmp{T}{C}\ table)\ s \\
    &\cmp{S}{drop table I}s &&= schemaChange\ \mathtt{I}\ \bot\ s \\
    &\cmp{S}{alter table I1 add I2 Ty}s &&= schemaChange\ \mathtt{I1}\ (\cmp{T}{\mathtt{I2} Ty}\ (s\ \mathtt{I1}))\ s\\
    &\cmp{S}{alter table I1 drop I2}s &&=
      schemaChange\ \mathtt{I1}\ (tableChange\ \mathtt{I2}\ \bot\ (\mathtt{s}\ \mathtt{I1}))\ s \\
    \\
    &\mathcal{T} : Declaration -> Table -> Table && \\
    &\cmp{T}{C1, C2}t &&= \cmp{T}{C2}\ (\cmp{T}{C1}\ t) \\
    &\cmp{T}{I Ty}t &&= tableChange\ \mathtt{I}\ \mathtt{Ty}\ t \\
  \end{alignat*}
  \normalsize
\end{frame}

\begin{frame}[fragile]{DDL Semantics}{Reduced Grammar}
  \begin{grammar}
    <Program> := <Declaration>;

    <Declaration> := <Declaration>; <Declaration> | <Create> | <Alter> | <Drop>

    <Create> := \lit*{create table} <Identifier> ( <Column> )

    <Column> := <Column>, <Column> | <Identifier> <Type>

    <Type> := \lit*{int} | \lit*{varchar(10)}
  \end{grammar}
\end{frame}

\begin{frame}[fragile]{DDL Semantics}{Denotational Semantics}
  \footnotesize
  \begin{alignat*}{2}
  &\delta \in \{-, +, \eta \} &&\\
  &\mathcal{S} : Differential -> Schema -> &&Schema \\
  &\cmp{S}{D;}^{\delta}s &&= \cmp{T}{D}^{\delta}\ s \\
  &\cmp{S}{D1;D2}^{\delta}s &&= \cmp{T}{D2}^{\delta}\ (\cmp{T}{D1}^{\delta}\ s) \\
  &\cmp{S}{create table I ( C )}^{-}s &&= schemaChange\ \mathtt{I}\ \bot\ s\\
  &\cmp{S}{create table I ( C )}^{+}s &&= schemaChange\ \mathtt{I}\ (\cmp{T}{C}\ table) \ s \\
  &\cmp{S}{create table I ( C )}^{\eta}s &&= schemaChange\ \mathtt{I}\ (\cmp{T}{C}\ (s\ \mathtt{I})) \\
\\
  &\mathcal{T} : Differential -> Table ->&& Table\\
  &\cmp{T}{C1, C2}^{\delta}t &&= \cmp{T}{C2}^{\delta}\ (\cmp{T}{C1}^{\delta}\ t) \\
  &\cmp{T}{I Ty}^{-}t &&= tableChange\ \mathtt{I}\ \bot\ t\\
  &\cmp{T}{I Ty}^{+}t &&= tableChange\ \mathtt{I}\ \mathtt{Ty}\ t\\
  &\cmp{T}{I Ty}^{\eta}t &&= t \\
  \end{alignat*}
  \normalsize
\end{frame}

\begin{frame}[fragile]
  \frametitle{Dynamic Software Update}
  \begin{block}{Type mismatch}
    Depending on the timing of an update an adapter may be required.
  \end{block}

  \begin{example}
    \begin{center}
      \begin{minipage}{.54\textwidth}
        \begin{lstlisting}[language=python]
# old version
# (Int, Int) -> String
def op(n, m):
 return str(sqrt(n, m))
        \end{lstlisting}
      \end{minipage}
      \hfill
      \begin{minipage}{.45\textwidth}
        \begin{lstlisting}[language=python]
# new version
# (Int, Int) -> Int
def op(n, m):
  return sqrt(n, m)
        \end{lstlisting}
      \end{minipage}
    \end{center}
  \end{example}
\end{frame}

\begin{frame}[fragile]
  \frametitle{Dynamic Software Update}
  \begin{block}{Type mismatch}
    The diff suggests \verb|str| as the adapter.
  \end{block}

  \begin{example}
    \lstinputlisting[language=diff,identifierstyle=\color{black}]{code/op.diff}
  \end{example}
\end{frame}

\end{document}
